\documentclass[a4paper,UKenglish,cleveref, autoref, thm-restate,authorcolumns]{lipics-v2019}

\usepackage[utf8]{inputenc}
\usepackage{url}
\usepackage{todonotes}
\usepackage{enumitem}
\setlist{parsep=0pt,listparindent=\parindent}
\usepackage{amssymb}
\usepackage{amsmath}
\usepackage[many]{tcolorbox}
\bibliographystyle{plainurl}

%%%%%%%%%%%%%%%%%%%%%%%%%%%%%%%%%%%%%%%%%%%%%%%%%%%%%%%%%%%%%%%%%%%
\title{Typing a linear \picalc}
\author{Uma Zalakain}{University of Glasgow, Scotland}
       {uma.zalakain.1@research.gla.ac.uk}{https://orcid.org/0000-0002-3268-9338}{}
\author{Ornela Dardha}{University of Glasgow, Scotland}
       {ornela.dardha@glasgow.ac.uk}{https://orcid.org/0000-0001-9927-7875}{}
\authorrunning{U. Zalakain and O. Dardha}
\Copyright{Uma Zalakain and Ornela Dardha}
\begin{CCSXML}
<ccs2012>
<concept>
<concept_id>10003752.10003753.10003761.10003764</concept_id>
<concept_desc>Theory of computation~Process calculi</concept_desc>
<concept_significance>300</concept_significance>
</concept>
</ccs2012>
\end{CCSXML}
\ccsdesc[300]{Theory of computation~Process calculi}
\keywords{pi calculus, linear, types, concurrency}
\supplement{\url{https://github/umazalakain/typing-linear-pi}}
\acknowledgements{I want to thank \dots}
%%%%%%%%%%%%%%%%%%%%%%%%%%%%%%%%%%%%%%%%%%%%%%%%%%%%%%%%%%%%%%%%%%%

% Commands for the pi-calculus
\newcommand{\PO}{\mathbf{0}}
\newcommand{\comp}[2]{#1 \mid #2}
\newcommand{\new}[2]{(\boldsymbol{\nu} #1 #2)\,}
\newcommand{\cout}[2]{\overline{#1}\langle#2\rangle.}
\newcommand{\cin}[2]{#1(#2).}
\newcommand{\select}[2]{#1\triangleleft#2.}
\newcommand{\branch}[2]{#1\triangleright#2}

\newcommand{\subst}[3]{#1[#2/#3]}

\newcommand{\picalc}{$\pi$-calculus}
\newcommand{\Picalc}{$\pi$-Calculus}

\newcommand{\type}{\texttt}
\newcommand{\End}{\type{End}}
\newcommand{\Send}[1]{!#1.}
\newcommand{\Recv}[1]{?#1.}
\newcommand{\Select}{\oplus}
\newcommand{\Branch}{\&}
\newcommand{\dual}{\overline}

\newcommand{\reduce}{\rightarrow}

\begin{document}

\maketitle

\begin{abstract}
  We present the syntax, operational semantics, and typing rules for a \picalc{} with linear and shared types. We use leftover typing \cite{} to encode our typing rules. We build framing and weakening proofs that we then use to prove subject congruence and subject reduction.
\end{abstract}

%%%%%%%%%%%%%%%%%%%%%%%%%%%%%%%%%%%%%%%%%%%%%%%%%%%%%%%%%%%%%%%%%%%
\section{Introduction}

Intensional (correct by construction) vs extensional

same syntax and semantics, different typing rules

\subsection{Contribution}

machine verified formalisation of the pi calculus

typing with leftovers applied to the pi calculus

unique quirks: multiple multiplicities per variable, shared types

full formalisation available in Agda

%%%%%%%%%%%%%%%%%%%%%%%%%%%%%%%%%%%%%%%%%%%%%%%%%%%%%%%%%%%%%%%%%%%
\section{Related work}

\cite{previous-work} polymorphic tokens, HOAS

\cite{typing-with-leftovers}

%%%%%%%%%%%%%%%%%%%%%%%%%%%%%%%%%%%%%%%%%%%%%%%%%%%%%%%%%%%%%%%%%%%
\section{Syntax}

variable references (strings, locally named, de Bruijn)

strings to maybe de Bruijn

%%%%%%%%%%%%%%%%%%%%%%%%%%%%%%%%%%%%%%%%%%%%%%%%%%%%%%%%%%%%%%%%%%%
\section{Semantics}

keeping track of the variable on which communication occurs

%%%%%%%%%%%%%%%%%%%%%%%%%%%%%%%%%%%%%%%%%%%%%%%%%%%%%%%%%%%%%%%%%%%
\section{Linear typing rules}

\subsection{Variable references}

\begin{description}
  \item[Polarities]
  \item[Multiple variables]
  \item[Vectors]
    Most general solution, $n$ possible multiplicities
\end{description}

\subsection{Multiplicities}

$\inf$ and $\mathbb{N}$ as generalisations

properties of the underlaying monoid

\subsubsection{Total definition}

\subsubsection{Partial definition}

partial vs total

\subsection{Variable types and multiplicities}

two-layered approach: types on one hand, capabilities on the other
removing from context vs keeping in context but marking it used

\subsection{Typing with leftovers}

\subsubsection{Typing relation}

Variable references as proofs of capability

Context splits at each variable reference

%%%%%%%%%%%%%%%%%%%%%%%%%%%%%%%%%%%%%%%%%%%%%%%%%%%%%%%%%%%%%%%%%%%
\section{Subject reduction}

\subsection{Framing}

Definition
Let $\Gamma \type P \boxtimes \Delta$ and $\Gamma - Delta \equiv \Xi - \phi$.
Then $\Xi \type P \boxtimes \phi$.

By defining of $\uplus$ as a total function, $\Gamma - \Delta$ is no longer
functional: $\omega - \omega$ results in any multiplicity, including $0$. Therefore
$\Xi \type P \boxtimes \phi$ would imply that such a variable cannot appear in $P$.

\subsection{Weakening}

%%%%%%%%%%%%%%%%%%%%%%%%%%%%%%%%%%%%%%%%%%%%%%%%%%%%%%%%%%%%%%%%%%%
\section{Future work}

Work that will be done time permiting:

Decidable typechecking

Proof of progress

Product types

Sum types

Encoding of session types


\bibliography{paper}
\end{document}
