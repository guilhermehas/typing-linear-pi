%% Appendix

\section{From names to de Bruijn indices and back}
\label{from_to_deBruijn}

\begin{figure}[h]
  \begin{mathpar}
    \datatype
    { }
    {\Raw : \Set}
    \; \textsc{Raw}
  \end{mathpar}

  \begin{equation*}
    \begin{aligned}
      \Raw ::= &\; \PO              &&\text{(inaction)}    \\ 
      |& \; (\new{\Name}) \; \Raw         &&\text{(restriction)} \\ 
      |& \; \comp{\Raw}{\Raw}       &&\text{(parallel)}    \\ 
      |& \; \recv{\Name}{\Name}\Raw &&\text{(input)}       \\ 
      |& \; \send{\Name}{\Name}\Raw &&\text{(output)}      \\
    \end{aligned}
  \end{equation*}
  \caption{Grammar with names.}
  \label{fig:syntax-names}
\end{figure}

The syntax of the \picalc{} \cite{Sangio01} using channel names is given by the $\Raw$ grammar in \autoref{fig:syntax-names}.

Channel names and variables range over $x,y,z$ in $\Name$ and processes over $P,Q,R$ in $\Raw$.
Process $\PO$ denotes the terminated process, where no further communications can occur.
Process $(\new{}x)\; P$ creates a new channel $x$ bound with scope $P$.
Process $\comp{P}{Q}$ is the parallel composition of processes $P$ and $Q$.
Processes $\recv{x}{y} P$ and $\send{x}{y} P$ denote respectively, the input and output processes of a variable $y$ over a channel $x$, with continuation $P$.
Scope restriction $(\new{}x) \; P$ and input $\recv{x}{y} \; P$ are \emph{binders}, they are the only constructs that introduce bound names --- $x$ and $y$ in $P$, respectively.

In order to demonstrate the correspondence between an \picalc{} that uses names and one that uses de Bruijn indices, we provide conversion functions in both directions and prove that they are inverses of each other up to $\alpha$-conversion.

\paragraph*{From names to de Bruijn indices}
When we translate into de Bruijn indices we keep the original binder names around --- they will serve as name hints for when we translate back.
The translation function $\func{fromRaw}$ works recursively, keeping a context $ctx : \Names_n$ that maps the first $n$ indices to their names.
Named references within the process are substituted with their corresponding de Bruijn index.
We demand that the original process is well-scoped: that all its free variable names appear in $ctx$ --- this is decidable and we therefore automate the construction of such a proof term.
\begin{alignat*}{2}
    &\func{fromRaw} && : (ctx : \Names_n) \; (P : \Raw) \\
    &               && \to \WellScoped \; ctx \; P \to \Process_n
\end{alignat*}

\paragraph*{From de Bruijn indices to names}
The translation function $\func{toRaw}$ works recursively, keeping a context $ctx : \Names_n$ that maps the first $n$ indices to their names.
As some widely-used languages do, this translation function produces unique variable names.
These unique variable names use the naming scheme $<name-hint>^{<n>}$, where $^{<n>}$ denotes that the name $<name-hint>$ has already been bound $n$ times before.
\begin{equation*}
    \func{toRaw} : (ctx : \Names_n) \to \Process_n \to \Raw
\end{equation*}

\begin{example}[$\func{fromRaw}$ and $\func{toRaw}$]
  \begin{figure}[t]
    \begin{alignat*}{7}
      &P = (\new{x} ) && (\comp {\recv{x}{x} && \send{x}{z} && \PO} {(\new{y}) && (\send{x}{y} && \recv{y}{y} && \PO)}) \\
      &Q = \new{} && (\comp {\recv{0}{} && \send{0}{2} && \PO} {\new{} && (\send{1}{0} && \recv{0}{} && \PO)}) \\
      &R = (\new{x^0} ) && (\comp {\recv{x^0}{x^1} && \send{x^1}{z^0} && \PO} {(\new{y^0}) && (\send{x^0}{y^0} && \recv{y^0}{y^1} && \PO)})
    \end{alignat*}
    \caption{From names to de Bruijn indices and back}
    \label{fig:conversion}
    \end{figure}
  We illustrate the conversion functions from names to de Bruijn indices $(\func{fromRaw}$) and back ($\func{toRaw}$) with three processes $P,Q,R$  in \autoref{fig:conversion}.

  Process $P$ uses names $x,y,z$ and is translated via the conversion function $\func{fromRaw}$ into process $Q$, which uses de Bruijn indices.
  Process $Q$ is then translated via $\func{toRaw}$ into process $R$, which follows the \emph{Barendregt convention}\footnote{The Barendregt variable convention states that all bound variables/names in a process are distinct among each other and from the free variables/names.} and is $\alpha$-equivalent to the original process $P$.
\end{example}

In the following we present the main results that our conversion functions satisfy.

\begin{lemma}
  Translating from de Bruijn indices to names via $\func{toRaw}$ results in a well-scoped process.
\end{lemma}

\begin{lemma}
  Translating from de Bruijn indices to names via $\func{toRaw}$ results in a process that follows the \emph{Barendregt convention}.
\end{lemma}

\begin{lemma}
  Translating from de Bruijn indices to names and back via $\func{fromRaw} \circ \func{toRaw}$ results in the same process modulo internal variable name hints.
\end{lemma}

\begin{lemma}
  Translating from names to de Bruijn indices and back via $\func{toRaw} \circ \func{fromRaw}$ results in the same process modulo $\alpha$-conversion.
\end{lemma}

\begin{proof}[Proof]
  All the above results are proved by induction on \textsc{Process} and \textsc{Var} (\autoref{fig:syntax}).
  Complete details can be found in our mechanisation in Agda in \cite{Zalakain2020Agda}.
\end{proof}
  
\section{Substitution}
\label{app:substitution-generalization}

\begin{theorem}[Generalised substitution]
  \label{thm:subst-generalization1}
  Let process $P$ be well-typed in $\types{\gamma}{\Gamma_i}{P}{\Psi_i}$.
  The substituted variable $i$ is capable of $m$ in $\Gamma_i$, and capable of $n$ in $\Psi_i$.
  Substitution will take these usages $m$ and $n$ away from $i$ and transfer them to the variable $j$ we are substituting for.
  In other words, there must exist some $\Gamma$, $\Psi$, $\Gamma_j$ and $\Psi_j$ such that:
  \begin{multicols}{2}
  \begin{itemize}
    \item $\contains{\gamma}{\Gamma_i}{i}{t}{m}{\Gamma}$
    \item $\contains{\gamma}{\Gamma_j}{j}{t}{m}{\Gamma}$
    \item $\contains{\gamma}{\Psi_i}{i}{t}{n}{\Psi}$
    \item $\contains{\gamma}{\Psi_j}{j}{t}{n}{\Psi}$
  \end{itemize}
  \end{multicols}
  Let $\Gamma$ and $\Psi$ be related such that $\opctx{\Gamma}{\Delta}{\Psi}$ for some $\Delta$.
  Let $\Delta$ have a usage annotation $\lz$ at position $i$, so that all consumption from $m$ to $n$ must happen in $P$.
  Then substituting $i$ to $j$ in $P$ will be well-typed in $\types{\gamma}{\Gamma_j}{\subst{P}{j}{i}}{\Psi_j}$.
  Refer to \autoref{fig:subst} for a diagramatic representation.
\end{theorem}

\begin{proof}[Proof]
  By induction on the derivation $\types{\gamma}{\Gamma_i}{P}{\Psi_i}$.
  \begin{itemize}
    \item
      For constructor $\PO$ we get $\Gamma_i \equiv \Psi_i$.
      From $\Delta_i \equiv \lz$ follows that $m \equiv n$.
      Therefore $\Gamma_j \equiv \Psi_j$ and $\constr{end}$ can be applied.

    \item
      For constructor $\new$ we proceed inductively, wrapping arrows $\ni_i m$, $\ni_j m$, $\ni_i n$ and $\ni_j n$ with $\suc$.
      
    \item
      For constructor $\recv{}{}$ we must split $\Delta$ to proceed inductively on the continuation.
      Observe that given the arrow from $\Gamma_i$ to $\Psi_i$ and given that $\Delta$ is $\lz$ at index $i$, there must exist some $\delta$ such that $\opsquared{m}{\delta}{n}$.
 l     \begin{itemize}
        \item
          If the input is on the variable being substituted, we split $m$ such that $\opsquared{m}{\li}{l}$ for some $l$, and construct an arrow $\containsusage{\Xi_i}{i}{l}{\Gamma}$ for the inductive call.
          Similarly, we construct for some $\Xi_j$ the arrows $\containsusage{\Gamma_j}{j}{\li}{\Xi_j}$ as the new input channel, and $\containsusage{\Xi_j}{j}{l}{\Gamma}$ for the inductive call.
        \item
          If the input is on a variable $x$ other than the one being substituted, we construct the arrows $\containsusage{\Xi_i}{i}{m}{\Theta}$ (for the inductive call) and $\containsusage{\Gamma}{x}{\li}{\Theta}$ for some $\Theta$.
          We then construct for some $\Xi_j$ the arrows $\containsusage{\Gamma_j}{x}{\li}{\Xi_j}$ (the new output channel) and $\containsusage{Xi_j}{j}{m}{\Theta}$ (for the inductive call).
          Given there exists a composition of arrows from $\Xi_i$ to $\Psi$, we conclude that $\Theta$ splits $\Delta$ such that $\opctx{\Gamma}{\Delta_1}{\Theta}$ and $\opctx{\Theta}{\Delta_2}{\Psi}$.
          As $\lz$ is a minimal element, then $\Delta_1$ must be $\lz$ at index $i$, and so must $\Delta_2$.
      \end{itemize}

    \item
      $\send{}{}$ applies the ideas outlined for the $\recv{}{}$ constructor to both the \textsc{VarRef} doing the output, and the \textsc{VarRef} for the sent data.

    \item
      For $\comp{}{}$ we first find a $\delta$, $\Theta$, $\Delta_1$ and $\Delta_2$ such that $\containsusage{\Xi_i}{i}{\delta}{\Theta}$ and $\opctx{\Gamma}{\Delta_1}{\Theta}$ and $\opctx{\Theta}{\Delta_2}{\Psi}$.
      Given $\Delta$ is $\lz$ at index $i$, we conclude that $\Delta_1$ and $\Delta_2$ are too.
      Observe that $\opsquared{m}{\delta}{\psi}$, where $\psi$ is the usage annotation at index $i$ consumed by the subprocess $P$.
      We construct an arrow $\containsusage{\Xi_j}{j}{\delta}{\Theta}$, for some $\Xi_j$.
      We can now make two inductive calls (on the derivation of $P$ and $Q$) and compose their results.
  \end{itemize}  
\end{proof}