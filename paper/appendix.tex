%% Appendix
\section{Lemmas}
\label{app:lemmas}

\begin{nilemma}
  \label{lm:renaming-unused}
  For every variables $i$ and $j$, if $i \not\equiv j$ then $\Unused_i (\subst{P}{j}{i})$.
\end{nilemma}
\begin{proof}
  By structural induction on \textsc{Process} and \textsc{Var}.
\end{proof}

\begin{nilemma}
  \label{lm:lower-lift}
  The function $\func{lower}_i \; P \; uP$ has an inverse $\func{lift}_i \; P$ that increments every $\textsc{Var}$ greater than or equal to $i$, such that $\func{lift}_i \; (\func{lower}_i \; P \; uP) \equiv P$.
\end{nilemma}
\begin{proof}
  By structural induction on \textsc{Process} and \textsc{Var}.
\end{proof}

\begin{nilemma}
  \label{lm:exchange-exchange}
  The function $\func{exchange}_i \; P$ is its own inverse: $\func{exchange}_i \; (\func{exchange}_i \; P) \equiv P$.
\end{nilemma}
\begin{proof}
  By structural induction on \textsc{Process} and \textsc{Var}.
\end{proof}

\begin{nilemma}
  \label{lm:types-unused}
  For all well-typed processes $\types{\gamma}{\Gamma}{P}{\Xi}$, if the variable $i$ is unused within $P$, then $\Gamma$ at $i$ is equivalent to $\Xi$ at $i$.
\end{nilemma}
\begin{proof}
  By induction on \textsc{Process} and \textsc{Var}.
\end{proof}

\begin{nilemma}
  \label{lm:comm-capable}
  Every input usage context $\Gamma$ of a well-typed process $\types{\gamma}{\Gamma}{P}{\Delta}$ that reduces by communicating on a channel external to it (that is, $P \reduce{\constr{external} \; i} Q$ for some $Q$) has a multiplicity of at least $\lio$ at index $i$.
\end{nilemma}

\begin{proof}
  By induction on the reduction derivation $P \reduce{\constr{external \; i}}Q$.
\end{proof}

\section{Structural Congruence}
\label{app:struct}

Structural congruence is a congruent equivalence relation.
As such, rewrites can happen anywhere inside a process, and are closed under reflexivity, symmetry and transitivity as shown by the first row of \autoref{fig:struct-cong1}.
The rest of the rules defines structural congruence under a context $\mathcal{C}[\cdot]$ \cite{Sangio01}, respectively restriction, composition, input and output.

\begin{figure}[h]
  \begin{mathpar}
    \datatype
    { }
    {\Rec : \Set}
    \; \textsc{Rec}
  
    \inferrule
    { }
    {\constr{zero} : \Rec}
    
    \inferrule
    {r : \Rec}
    {\constr{one} \; r : \Rec}
  
    \inferrule
    {r \; s : \Rec}
    {\constr{two} \; r \; s : \Rec}
    
    \datatype
    {P \, Q : \Process_n \\ r : \Rec}
    {P \eq{r} Q : \Set}
    \; \textsc{Equals}
  
    \inferrule
    {eq : P \eqeq Q}
    {\constr{struct} \; eq : P \eq{\constr{zero}} Q}
  
    \inferrule
    {eq : P \eq{r} P'}
    {\constr{cong-scope} \; eq : \new P \eq{\constr{one} \; r} \new P'}
  
    \inferrule
    {eq : P \eq{r} P'}
    {\constr{cong-comp} \; eq : \comp{P}{Q} \eq{\constr{one} \; r} \comp{P'}{Q}}
  
    \inferrule
    {eq : P \eq{r} P'}
    {\constr{cong-recv} \; eq : \recv{x}P \eq{\constr{one} \; r} \recv{x}P'}
  
    \inferrule
    {eq : P \eq{r} P'}
    {\constr{cong-send} \; eq : \send{x}{y}P \eq{\constr{one} \; r} \send{x}{y}P'}
  
    \inferrule
    { }
    {\constr{refl} : P \eq{\constr{zero}} P}
  
    \inferrule
    {eq : P \eq{r} Q}
    {\constr{sym} \; eq : Q \eq{\constr{one} \; r} P}
  
    \inferrule
    {eq_1 : P \eq{r} Q \\ \; eq_2 : Q \eq{s} R}
    {\constr{trans} \; eq_1 \; eq_2 : P \eq{\constr{two} \; r \; s} R}
  \end{mathpar}
  \caption{Structural rewriting rules lifted to a congruent equivalence relation indexed by a recursion tree.}
  \label{fig:struct-cong1}
  \end{figure}

In the transitivity rule, we must show that if $P$ is structurally congruent to $Q$ and $Q$ is to $R$, and $P$ is well-typed, then so is $R$.
To do so, we need to proceed by induction and first get a proof of the well-typedness of $Q$, then use that to reach $R$.
To show the typechecker that the doubly recursive call terminates we index the equivalence relation $=$ by a type $\Rec$ that models the structure of the recursion.

Subject congruence states that applying structural congruence rules to a well typed process preserves its well-typedness.
\begin{nitheorem}[Subject congruence]
  \label{thm:subject-congruence1}
  If $P \eq{} Q$ and $\types{\gamma}{\Gamma}{P}{\Xi}$, then $\types{\gamma}{\Gamma}{Q}{\Xi}$.
\end{nitheorem}

\begin{proof}[Proof (Sketch)]
  The proof is by induction on \textsc{Equals} $\eq{}$.
  Here we only consider the nontrivial base cases for $\constr{struct}$ and their symmetric variants.
  Full proof in \cite{Zalakain2020Agda}.
  We proceed by induction on \textsc{StructCong} $\eqeq$:
  \begin{itemize}
    \item
      Case $\constr{comp-assoc}$: trivial, as leftover typing is naturally associative.
    \item
      Case $\constr{comp-sym}$ for $\comp{P}{Q}$: we use framing (\autoref{thm:framing}) to shift the output context of $P$ to the one of $Q$; and the input context of $Q$ to the one of $P$.
    \item
      Case $\constr{comp-end}$: trivial, as the typing rule for $\PO$ has the same input and output contexts.
    \item
      Case $\constr{scope-end}$: we show that the usage annotation of the newly created channel must be $\lz$, making the proof trivial.
      In the opposite direction, we instantiate the newly created channel to a type $\unit$ and a usage annotation $\lz$..
    \item
      Case $\constr{scope-ext}$ for $\new \comp{P}{Q}$: we to show that $P$ preserves the usage annotations of the unused variable (\autoref{lm:types-unused}) and then use strengthening (\autoref{thm:strengthening}).
      In the reverse direction, we use weakening (\autoref{thm:weakening}) on $P$ and show that lowering and then lifting $P$ results in $P$ (\autoref{lm:lower-lift}).
    \item
      Case $\constr{scope-comm}$:
      we use exchange (\autoref{thm:exchange}), and for the reverse direction exchange and \autoref{lm:exchange-exchange} to show that exchanging two elements in $P$ twice leaves $P$ unchanged. \qedhere
  \end{itemize}
\end{proof}

\section{Renaming}
\label{app:renaming-generalization}

\begin{nitheorem}[Generalised renaming]
  \label{thm:subst-generalization1}
  Let process $P$ be well-typed in $\types{\gamma}{\Gamma_i}{P}{\Psi_i}$.
  The substituted variable $i$ is capable of $m$ in $\Gamma_i$, and capable of $n$ in $\Psi_i$.
  Renaming will take these usages $m$ and $n$ away from $i$ and transfer them to the variable $j$ we are substituting for.
  In other words, there must exist some $\Gamma$, $\Psi$, $\Gamma_j$ and $\Psi_j$ such that:
  \begin{multicols}{2}
  \begin{itemize}
    \item $\contains{\gamma}{\Gamma_i}{i}{t}{m}{\Gamma}$
    \item $\contains{\gamma}{\Gamma_j}{j}{t}{m}{\Gamma}$
    \item $\contains{\gamma}{\Psi_i}{i}{t}{n}{\Psi}$
    \item $\contains{\gamma}{\Psi_j}{j}{t}{n}{\Psi}$
  \end{itemize}
  \end{multicols}
  Let $\Gamma$ and $\Psi$ be related such that $\opctx{\Gamma}{\Delta}{\Psi}$ for some $\Delta$.
  Let $\Delta$ have a usage annotation $\lz$ at position $i$, so that all consumption from $m$ to $n$ must happen in $P$.
  Then renaming $i$ to $j$ in $P$ will be well-typed in $\types{\gamma}{\Gamma_j}{\subst{P}{j}{i}}{\Psi_j}$.
  Refer to \autoref{fig:subst} for a diagramatic representation.
\end{nitheorem}

\begin{proof}[Proof (Sketch)]
  By induction on the derivation $\types{\gamma}{\Gamma_i}{P}{\Psi_i}$.
  \begin{itemize}
    \item
      For constructor $\PO$ we get $\Gamma_i \equiv \Psi_i$.
      From $\Delta_i \equiv \lz$ follows that $m \equiv n$.
      Therefore $\Gamma_j \equiv \Psi_j$ and $\constr{end}$ can be applied.

    \item
      For constructor $\new$ we proceed inductively, wrapping arrows $\ni_i m$, $\ni_j m$, $\ni_i n$ and $\ni_j n$ with $\suc$.
      
    \item
      For constructor $\recv{}{}$ we must split $\Delta$ to proceed inductively on the continuation.
      Observe that given the arrow from $\Gamma_i$ to $\Psi_i$ and given that $\Delta$ is $\lz$ at index $i$, there must exist some $\delta$ such that $\opsquared{m}{\delta}{n}$.
 l     \begin{itemize}
        \item
          If the input is on the variable being substituted, we split $m$ such that $\opsquared{m}{\li}{l}$ for some $l$, and construct an arrow $\containsusage{\Xi_i}{i}{l}{\Gamma}$ for the inductive call.
          Similarly, we construct for some $\Xi_j$ the arrows $\containsusage{\Gamma_j}{j}{\li}{\Xi_j}$ as the new input channel, and $\containsusage{\Xi_j}{j}{l}{\Gamma}$ for the inductive call.
        \item
          If the input is on a variable $x$ other than the one being substituted, we construct the arrows $\containsusage{\Xi_i}{i}{m}{\Theta}$ (for the inductive call) and $\containsusage{\Gamma}{x}{\li}{\Theta}$ for some $\Theta$.
          We then construct for some $\Xi_j$ the arrows $\containsusage{\Gamma_j}{x}{\li}{\Xi_j}$ (the new output channel) and $\containsusage{Xi_j}{j}{m}{\Theta}$ (for the inductive call).
          Given there exists a composition of arrows from $\Xi_i$ to $\Psi$, we conclude that $\Theta$ splits $\Delta$ such that $\opctx{\Gamma}{\Delta_1}{\Theta}$ and $\opctx{\Theta}{\Delta_2}{\Psi}$.
          As $\lz$ is a minimal element, then $\Delta_1$ must be $\lz$ at index $i$, and so must $\Delta_2$.
      \end{itemize}

    \item
      $\send{}{}$ applies the ideas outlined for the $\recv{}{}$ constructor to both the \textsc{VarRef} doing the output, and the \textsc{VarRef} for the sent data.

    \item
      For $\comp{}{}$ we first find a $\delta$, $\Theta$, $\Delta_1$ and $\Delta_2$ such that $\containsusage{\Xi_i}{i}{\delta}{\Theta}$ and $\opctx{\Gamma}{\Delta_1}{\Theta}$ and $\opctx{\Theta}{\Delta_2}{\Psi}$.
      Given $\Delta$ is $\lz$ at index $i$, we conclude that $\Delta_1$ and $\Delta_2$ are too.
      Observe that $\opsquared{m}{\delta}{\psi}$, where $\psi$ is the usage annotation at index $i$ consumed by the subprocess $P$.
      We construct an arrow $\containsusage{\Xi_j}{j}{\delta}{\Theta}$, for some $\Xi_j$.
      We can now make two inductive calls (on the derivation of $P$ and $Q$) and compose their results.
  \end{itemize}  
\end{proof}