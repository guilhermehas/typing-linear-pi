%% Appendix
\section{Structural Congruence}
\label{app:struct}

Structural congruence is a congruent equivalence relation.
As such, rewrites can happen anywhere inside a process, and are closed under reflexivity, symmetry and transitivity as shown by the first row of \autoref{fig:struct-cong1}.
The rest of the rules defines structural congruence under a context $\mathcal{C}[\cdot]$ \cite{Sangio01}, respectively restriction, composition, input and output.

\begin{figure}[h]
  \begin{mathpar}
    \datatype
    { }
    {\Rec : \Set}
    \; \textsc{Rec}
  
    \inferrule
    { }
    {\constr{zero} : \Rec}
    
    \inferrule
    {r : \Rec}
    {\constr{one} \; r : \Rec}
  
    \inferrule
    {r \; s : \Rec}
    {\constr{two} \; r \; s : \Rec}
    
    \datatype
    {P \, Q : \Process_n \\ r : \Rec}
    {P \eq{r} Q : \Set}
    \; \textsc{Equals}
  
    \inferrule
    {eq : P \eqeq Q}
    {\constr{struct} \; eq : P \eq{\constr{zero}} Q}
  
    \inferrule
    {eq : P \eq{r} P'}
    {\constr{cong-scope} \; eq : \new P \eq{\constr{one} \; r} \new P'}
  
    \inferrule
    {eq : P \eq{r} P'}
    {\constr{cong-comp} \; eq : \comp{P}{Q} \eq{\constr{one} \; r} \comp{P'}{Q}}
  
    \inferrule
    {eq : P \eq{r} P'}
    {\constr{cong-recv} \; eq : \recv{x}P \eq{\constr{one} \; r} \recv{x}P'}
  
    \inferrule
    {eq : P \eq{r} P'}
    {\constr{cong-send} \; eq : \send{x}{y}P \eq{\constr{one} \; r} \send{x}{y}P'}
  
    \inferrule
    { }
    {\constr{refl} : P \eq{\constr{zero}} P}
  
    \inferrule
    {eq : P \eq{r} Q}
    {\constr{sym} \; eq : Q \eq{\constr{one} \; r} P}
  
    \inferrule
    {eq_1 : P \eq{r} Q \\ \; eq_2 : Q \eq{s} R}
    {\constr{trans} \; eq_1 \; eq_2 : P \eq{\constr{two} \; r \; s} R}
  \end{mathpar}
  \caption{Structural rewriting rules lifted to a congruent equivalence relation indexed by a recursion tree.}
  \label{fig:struct-cong1}
  \end{figure}

In the transitivity rule, we must show that if $P$ is structurally congruent to $Q$ and $Q$ is to $R$, and $P$ is well-typed, then so is $R$.
To do so, we need to proceed by induction and first get a proof of the well-typedness of $Q$, then use that to reach $R$.
To show the typechecker that the doubly recursive call terminates we index the equivalence relation $=$ by a type $\Rec$ that models the structure of the recursion.

\section{Lemmas}
\label{app:lemmas}

\begin{nilemma}
  \label{lm:subst-unused}
  For every variables $i$ and $j$, if $i \not\equiv j$ then $\Unused_i (\subst{P}{j}{i})$.
\end{nilemma}
\begin{proof}
  By structural induction on \textsc{Process} and \textsc{Var}.
\end{proof}

\begin{nilemma}
  \label{lm:lower-lift}
  The function $\func{lower}_i \; P \; uP$ has an inverse $\func{lift}_i \; P$ that increments every $\textsc{Var}$ greater than or equal to $i$, such that $\func{lift}_i \; (\func{lower}_i \; P \; uP) \equiv P$.
\end{nilemma}
\begin{proof}
  By structural induction on \textsc{Process} and \textsc{Var}.
\end{proof}

\begin{nilemma}
  \label{lm:swap-swap}
  The function $\func{swap}_i \; P$ is its own inverse: $\func{swap}_i \; (\func{swap}_i \; P) \equiv P$.
\end{nilemma}
\begin{proof}
  By structural induction on \textsc{Process} and \textsc{Var}.
\end{proof}

\begin{nilemma}
  \label{lm:types-unused}
  For all well-typed processes $\types{\gamma}{\Gamma}{P}{\Xi}$, if the variable $i$ is unused within $P$, then $\Gamma$ at $i$ is equivalent to $\Xi$ at $i$.
\end{nilemma}
\begin{proof}
  By induction on \textsc{Process} and \textsc{Var}.
\end{proof}

\begin{nilemma}
  \label{lm:comm-capable}
  Every input usage context $\Gamma$ of a well-typed process $\types{\gamma}{\Gamma}{P}{\Delta}$ that reduces by communicating on a channel external to it (that is, $P \reduce{\constr{external} \; i} Q$ for some $Q$) has a multiplicity of at least $\lio$ at index $i$.
\end{nilemma}

\begin{proof}
  By induction on the reduction derivation $P \reduce{\constr{external \; i}}Q$.
\end{proof}